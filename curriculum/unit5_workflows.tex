%%%%%%%%%%%%%%%%%%%%%%%%%%%%%%%%%%%%%%%%%%%%%%%%%%%%%%%%%%%%%%%%%
\chapter{Unit 5: Reproducible Workflows}
\label{chap:unit5}
%%%%%%%%%%%%%%%%%%%%%%%%%%%%%%%%%%%%%%%%%%%%%%%%%%%%%%%%%%%%%%%%%

\textbf{Total Time: 5.5 hours}

%================================================================
\section{Goals of Reproducible Analyses}
\label{sec:5.1}
%================================================================

Learn the key goals and challenges of creating reproducible, transparent, and user-friendly analyses that are easy to share and reuse.

\subsection{Learning Objectives}

\begin{enumerate}
    \item Awareness of key challenges and goals when creating reproducible workflows, including making analyses reproducible, user friendly, transparent, reusable, version controlled, and archived.
\end{enumerate}

%================================================================
\section{Reproducibility via Code Notebooks}
\label{sec:5.2}
%================================================================

Gain awareness of Markdown, Jupyter, and Quarto, and learn how these tools integrate to create clear, reproducible workflows for data analysis and reporting.

\subsection{Learning Objectives}

\begin{enumerate}
    \item Awareness of Markdown, Jupyter, Quarto, and how these tools can be integrated into reproducible workflows.
\end{enumerate}

%================================================================
\section{Best Practices for Reproducible Programming}
\label{sec:5.3}
%================================================================

Learn essential best practices for reproducible programming, including writing clear scripts and functions, avoiding magic numbers, using caching and seeding for randomness, and refactoring code to enhance clarity, reliability, and repeatability.

\subsection{Learning Objectives}

\begin{enumerate}
    \item Awareness of best practices for reproducible programming including writing scripts, functions, avoiding magic numbers, caching and seeding randomness, and how to refactor code to align with these practices.
\end{enumerate}

%================================================================
\section{Version Control}
\label{sec:5.4}
%================================================================

Gain a basic understanding of Git, its advantages, and learn to perform essential tasks such as cloning repositories, committing changes, and syncing with remote repositories using push and pull commands.

\subsection{Learning Objectives}

\begin{enumerate}
    \item Familiarity with Git and its benefits, and the ability to begin using it for simple tasks, including cloning, committing changes, pushing and pulling.
\end{enumerate}

%================================================================
\section{Containers}
\label{sec:5.5}
%================================================================

Gain hands-on experience with key dependency management tools (Python virtual environments, renv, and containerization), understanding their pros and cons, and develop the skills to create and run basic Docker images.

\subsection{Learning Objectives}

\begin{enumerate}
    \item Familiarity with various tools for dependency management, including Python virtual environments, renv, and containerization, and their respective strengths and weaknesses. Ability to create and run simple Docker images.
\end{enumerate}

%================================================================
\section{Assembling a Full Analysis Pipeline}
\label{sec:5.6}
%================================================================

Learn key factors in organizing an analysis pipeline and develop the skills to assemble a complete, reusable pipeline template.

\subsection{Learning Objectives}

\begin{enumerate}
    \item Considerations when organizing an analysis pipeline, and the ability to assemble a full template pipeline.
\end{enumerate}

\subsection{Assessment Instrument}

\begin{enumerate}
    \item Describe your progress on the template workflow. What aspects did you find most confusing or challenging? Which tools (e.g., Git, Make, Docker) were hardest to implement, and why?
    \item What is one thing you plan to change or do differently in your own projects after today's session? Give a specific example of an analysis or workflow improvement you intend to make.
\end{enumerate}

% Required packages (add to your preamble if not already there):
% \usepackage{longtable}
% \usepackage{booktabs}
% \usepackage{ragged2e}
% \usepackage{array}
% \usepackage{pdflscape}

\begin{landscape}

    \section{Evaluation Rubric}

    {\footnotesize

    \begin{longtable}{%
        >{\RaggedRight\arraybackslash}p{2.5cm}   % Component
        >{\RaggedRight\arraybackslash}p{4.5cm}   % Excellent
        >{\RaggedRight\arraybackslash}p{4.2cm}   % Good
        >{\RaggedRight\arraybackslash}p{4.2cm}   % Satisfactory
        >{\RaggedRight\arraybackslash}p{4.5cm}}  % Needs Improvement

    \toprule
    \textbf{Component} &
    \textbf{Excellent (90--100\%)} &
    \textbf{Good (80--89\%)} &
    \textbf{Satisfactory (70--79\%)} &
    \textbf{Needs Improvement ($<$70\%)} \\
    \midrule
    \endfirsthead

    \toprule
    \textbf{Component} &
    \textbf{Excellent (90--100\%)} &
    \textbf{Good (80--89\%)} &
    \textbf{Satisfactory (70--79\%)} &
    \textbf{Needs Improvement ($<$70\%)} \\
    \midrule
    \endhead

    \midrule
    \multicolumn{5}{r}{\textit{Continued on next page}} \\
    \endfoot

    \bottomrule
    \endlastfoot

    % Row 1
    \textbf{1.\ Goals of Reproducible Analyses}
    \newline\textbf{(10\%)}
    &
    Provides a thorough summary, clearly articulating key goals/challenges of
    reproducible analyses; connects concepts to personal work and the CDC
    dataset; demonstrates careful reflection on transparency, usability,
    version control, and archiving.
    &
    Addresses most key challenges/goals clearly; provides some connection to
    personal work and dataset; demonstrates solid understanding with minor
    omissions.
    &
    Identifies basic challenges/goals, but with limited depth or reflection;
    may not connect well to examples or skip some elements.
    &
    Incomplete or superficial summary; lacks understanding of core concepts
    or fails to address relevance to own work.
    \\
    \midrule

    % Row 2
    \textbf{2.\ Reproducibility via Code Notebooks}
    \newline\textbf{(20\%)}
    &
    Creates a well-structured notebook with reproducible code, markdown, at
    least one plot and table; documentation is clear; script downloads CDC
    dataset; all work uploaded; demonstrates integration of tools.
    &
    Notebook includes most required elements (code, markdown, plot/table);
    documentation is mostly clear; dataset obtained; minor gaps in
    reproducibility or clarity.
    &
    Notebook has basic elements, but some required elements (plot/table,
    markdown, script) are missing or poorly integrated; minimal documentation.
    &
    Notebook missing key requirements; poorly documented or not reproducible;
    files not uploaded, or CDC dataset not included.
    \\
    \midrule

    % Row 3
    \textbf{3.\ Best Practices for Reproducible Programming}
    \newline\textbf{(15\%)}
    &
    Adds thorough documentation to notebook; creates a well-organized Makefile
    with clear, reusable commands; analysis file is uploaded; demonstrates
    strong understanding of reproducible scripting (functions, no magic
    numbers, caching, seeding randomness, refactoring).
    &
    Documentation and Makefile present and mostly clear; analysis file
    uploaded; most best practices followed, though minor omissions may be
    present.
    &
    Documentation or Makefile is incomplete or unclear; basic understanding
    of practices is evident, but with gaps in execution or clarity.
    &
    Documentation and/or Makefile missing or severely lacking; little
    evidence of understanding reproducible programming practices.
    \\
    \midrule

    % Row 4
    \textbf{4.\ Version Control}
    \newline\textbf{(15\%)}
    &
    Template project code on GitHub is complete and well-structured; the link
    is shared; meaningful commit history demonstrates staged progress and
    clarity; challenges are described in detail.
    &
    Project is on GitHub with most files complete; link is shared; some
    commit history and brief challenge discussion; mostly correct use of git.
    &
    Project present but incomplete or poorly organized; minimal commit
    history; challenges noted superficially.
    &
    Project not on GitHub or files missing; little/no evidence of version
    control understanding; challenges not described.
    \\
    \midrule

    % Row 5
    \textbf{5.\ Containers \& Dependency Management}
    \newline\textbf{(20\%)}
    &
    The template project is successfully containerized (Docker/renv/venv);
    runs as expected; Dockerfile/other files uploaded; demonstrates a clear
    understanding of strengths/weaknesses; GitHub project is complete and
    well-documented; challenges discussed.
    &
    Container image runs with minor issues; most files uploaded; reasonable
    documentation and explanation of tool choices; challenges discussed
    briefly.
    &
    Container created but may not run as expected, or files/documentation
    incomplete; basic understanding of tools, but lacks depth.
    &
    Container not created or does not run; missing files/documentation;
    little/no understanding of dependency management.
    \\
    \midrule

    % Row 6
    \textbf{6.\ Assembling a Full Analysis Pipeline}
    \newline\textbf{(15\%)}
    &
    Creates a comprehensive template project with a clear directory structure
    and markdown documentation; the pipeline demonstrates thoughtful
    organization, reproducibility, and ease of use; meets all requirements
    outlined in class with innovative or robust elements.
    &
    Project structure and markdown are mostly clear; most requirements met,
    with minor organizational or documentation gaps.
    &
    Basic directory structure and markdown present; pipeline missing some
    components or documentation unclear; meets minimal requirements.
    &
    Template project missing core structure or documentation; does not
    demonstrate pipeline assembly skills; requirements not met.
    \\
    \midrule

    % Row 7
    \textbf{7.\ Professionalism, Clarity, Structure}
    \newline\textbf{(5\%)}
    &
    Work is polished, well-organized, clear, and follows all instructions;
    files are named and formatted consistently; writing is succinct and
    logical.
    &
    Work is mostly clear and organized; minor formatting or organizational
    issues.
    &
    Basic clarity but contains organizational weaknesses, unclear writing, or
    some failures to follow instructions.
    &
    Work is disorganized, unclear, or does not follow instructions; formatting
    and naming are inconsistent or missing.
    \\

    \end{longtable}

    } % end \footnotesize

\end{landscape}