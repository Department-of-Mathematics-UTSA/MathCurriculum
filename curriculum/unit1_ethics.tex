%%%%%%%%%%%%%%%%%%%%%%%%%%%%%%%%%%%%%%%%%%%%%%%%%%%%%%%%%%%%%%%%%
\chapter{Unit 1: Responsible Conduct of Research}
\label{chap:unit1}
%%%%%%%%%%%%%%%%%%%%%%%%%%%%%%%%%%%%%%%%%%%%%%%%%%%%%%%%%%%%%%%%%

\textbf{Total Time: 3 hours}

%================================================================
\section{RCR in the Context of Biomedical Data Science}
\label{sec:1.1}
%================================================================

This lesson examines the sociotechnical and ethical aspects of biomedical data science. We will consider ethical issues in the responsible conduct of research that are novel to or pose new challenges in the context of biomedical data science such as reproducibility and privacy. Students will also consider biomedical data science as a sociotechnical system and define roles for themselves and other key constituents.

\subsection{Learning Objectives}

\begin{enumerate}
    \item Explain novel ethical issues in responsible conduct of research for data science such as reproducibility and privacy.
    \item Describe the landscape of biomedical data science as a sociotechnical system and articulate roles.
\end{enumerate}

\subsection{Assessment Instrument}

\begin{itemize}
    \item Compare responses to the data challenge with your peers. What issues arise?
    \item Why was date and time no longer recorded after 1987?
    \item Who decides what data to collect, how to store it and how to access it? What biases could there be in the data (e.g., data collection in rural areas, existence of infrastructure)? What is the difference between bias and trend in these data? Provide examples.
\end{itemize}

%================================================================
\section{What are Ethics? Ethical Issues in Biomedical Data Science}
\label{sec:1.2}
%================================================================

This lesson equips students to address ethical challenges in biomedical data science. Learners will identify strategies for ethical secondary data use, analyze engagement approaches, and develop frameworks for ethical project review, emphasizing anticipatory governance and responsible data science practices. Case studies will be used that draw on the group project selected for the 2026 cohort.

\subsection{Learning Objectives}

\begin{enumerate}
    \item Differentiate between traditional bioethical, sociotechnical, and other ethical approaches to data science research and applications.
    \item Evaluate key ethical challenges in biomedical data science.
    \item Identify and formulate approaches to address ethical issues in secondary use, including anticipatory governance principles.
    \item Develop a framework for ethical review of biomedical data science projects.
\end{enumerate}

\subsection{Assessment Instrument}

The assessment involves designing a governance framework for a data science initiative, selecting one of three projects derived from the course-wide data activity. The framework must address stakeholder engagement, decision-making, monitoring, benefit-sharing, unexpected impacts, and consent, while evaluating ethical considerations, future challenges, and feasibility. The goal is to create an ethical, well-structured, and adaptable governance plan.


% Required packages (add to your preamble if not already there):
% \usepackage{longtable}
% \usepackage{booktabs}
% \usepackage{ragged2e}
% \usepackage{array}
% \usepackage{pdflscape}

\begin{landscape}

    \section{Evaluation Rubric}

    {\footnotesize

    \begin{longtable}{%
        >{\RaggedRight\arraybackslash}p{2.5cm}   % Component
        >{\RaggedRight\arraybackslash}p{4.5cm}   % Excellent
        >{\RaggedRight\arraybackslash}p{4.2cm}   % Good
        >{\RaggedRight\arraybackslash}p{4.2cm}   % Satisfactory
        >{\RaggedRight\arraybackslash}p{4.5cm}}  % Needs Improvement

    \toprule
    \textbf{Component} &
    \textbf{Excellent (90--100\%)} &
    \textbf{Good (80--89\%)} &
    \textbf{Satisfactory (70--79\%)} &
    \textbf{Needs Improvement ($<$70\%)} \\
    \midrule
    \endfirsthead

    \toprule
    \textbf{Component} &
    \textbf{Excellent (90--100\%)} &
    \textbf{Good (80--89\%)} &
    \textbf{Satisfactory (70--79\%)} &
    \textbf{Needs Improvement ($<$70\%)} \\
    \midrule
    \endhead

    \midrule
    \multicolumn{5}{r}{\textit{Continued on next page}} \\
    \endfoot

    \bottomrule
    \endlastfoot

    % Row 1
    \textbf{1.\ Peer Comparison \& Issue Identification}
    \newline\textbf{(10\%)}
    &
    Thoughtfully compares responses with peers and identifies multiple
    substantive issues; demonstrates critical reflection on discrepancies
    and their ethical or scientific implications; connects issues to
    broader reproducibility and data quality concerns.
    &
    Compares responses and identifies several relevant issues; discussion
    is mostly substantive with some reflection on implications; connection
    to reproducibility or data quality is partially developed.
    &
    Identifies some issues from peer comparison but analysis is surface
    level; limited reflection on why discrepancies matter for data
    science practice.
    &
    Minimal or no meaningful peer comparison; issues identified are vague
    or incorrect; little evidence of critical engagement with the data
    challenge.
    \\
    \midrule

    % Row 2
    \textbf{2.\ Historical Data Context}
    \newline\textbf{(10\%)}
    &
    Provides a well-reasoned, nuanced explanation for why date and time
    recording ceased after 1987; situates the change within broader
    sociotechnical, institutional, or policy contexts; demonstrates
    understanding of how historical decisions shape data availability
    and reproducibility.
    &
    Offers a reasonable explanation with some contextual grounding;
    demonstrates solid understanding of how institutional or policy
    factors affect data collection practices, with minor gaps.
    &
    Provides a basic explanation but lacks contextual depth; recognizes
    that external factors influence data collection without fully
    analyzing their implications.
    &
    Explanation is missing, or purely speculative; no
    grounding in sociotechnical or historical context; shows little
    understanding of how data collection decisions are made.
    \\
    \midrule

    % Row 3
    \textbf{3.\ Data Governance, Bias, \& Trends}
    \newline\textbf{(15\%)}
    &
    Clearly articulates who decides what data to collect, store, and
    access; identifies multiple plausible sources of bias (e.g., rural
    infrastructure, collection gaps) with specific examples; provides a
    nuanced, accurate distinction between bias and trend supported by
    concrete examples; demonstrates sophisticated understanding of how
    governance decisions shape data quality and equity.
    &
    Identifies key decision-makers and discusses bias sources with
    reasonable examples; distinction between bias and trend is mostly
    clear and accurate; connection to equity or data quality mostly
    well developed.
    &
    Names some stakeholders but analysis is incomplete; identifies at
    least one source of bias with a limited example; distinction between
    bias and trend is attempted but imprecise or only partly illustrated.
    &
    Fails to address who governs data collection, storage, or access;
    bias sources missing or incorrectly described; no meaningful
    distinction between bias and trend; examples absent or irrelevant.
    \\
    \midrule

    % Row 4
    \textbf{4.\ Stakeholder Engagement \& Decision-Making}
    \newline\textbf{(15\%)}
    &
    Framework identifies all relevant stakeholders with well-justified
    roles; describes specific, realistic engagement strategies;
    articulates a clear and equitable decision-making structure;
    anticipates conflicts of interest and addresses them thoughtfully.
    &
    Most stakeholders identified with reasonable engagement approaches;
    decision-making structure is mostly clear; minor gaps in equity
    considerations or conflict of interest analysis.
    &
    Key stakeholders named but engagement strategies are generic or
    underdeveloped; decision-making structure is present but lacks
    clarity or justification.
    &
    Stakeholders missing or poorly identified; no coherent engagement
    strategy or decision-making structure; shows limited understanding
    of governance design.
    \\
    \midrule

    % Row 5
    \textbf{5.\ Ethical Considerations \& Anticipatory Governance}
    \newline\textbf{(20\%)}
    &
    Thoroughly evaluates ethical challenges specific to the chosen
    project; integrates anticipatory governance principles to address
    future uncertainties; differentiates among traditional bioethical,
    sociotechnical, and other frameworks with sophistication; analysis
    is proactive rather than reactive.
    &
    Ethical challenges well evaluated; anticipatory governance referenced
    and mostly applied; frameworks differentiated with minor inaccuracies
    or incomplete integration.
    &
    Some ethical challenges identified; anticipatory governance mentioned
    but superficially applied; framework differentiation is partial or
    inconsistent.
    &
    Ethical considerations minimal or inaccurate; no meaningful
    application of anticipatory governance; frameworks absent,
    conflated, or misapplied.
    \\
    \midrule

    % Row 6
    \textbf{6.\ Monitoring, Unexpected Impacts, \& Benefit-Sharing}
    \newline\textbf{(15\%)}
    &
    Proposes specific, feasible monitoring mechanisms with clear
    indicators; thoughtfully anticipates unexpected impacts and describes
    adaptive responses; benefit-sharing plan is equitable, well-reasoned,
    and tied to the project-specific context.
    &
    Monitoring and benefit-sharing addressed with reasonable detail;
    unexpected impacts considered with some adaptive strategies; minor
    gaps in feasibility or equity analysis.
    &
    Monitoring and benefit-sharing mentioned but lack specificity or
    justification; unexpected impacts acknowledged without substantive
    adaptive planning.
    &
    Monitoring, benefit-sharing, or unexpected impacts missing or poorly
    addressed; framework is reactive or purely theoretical with no
    practical grounding.
    \\
    \midrule

    % Row 7
    \textbf{7.\ Consent \& Secondary Data Use}
    \newline\textbf{(10\%)}
    &
    Consent framework is clearly articulated and appropriately tailored
    to secondary data use; identifies specific privacy and confidentiality
    risks; proposes concrete strategies consistent with ethical standards
    and relevant policy or regulatory requirements.
    &
    Consent addressed with reasonable detail; privacy risks identified;
    strategies are mostly appropriate with minor gaps in policy alignment
    or specificity.
    &
    Consent mentioned but treatment is generic or incompletely applied
    to the secondary use context; privacy risks acknowledged without
    substantive mitigation strategies.
    &
    Consent absent, incorrect, or irrelevant to secondary use; privacy
    concerns not addressed; no evidence of understanding relevant
    ethical or regulatory standards.
    \\
    \midrule

    % Row 8
    \textbf{8.\ Feasibility \& Overall Framework Quality}
    \newline\textbf{(5\%)}
    &
    Framework is coherent, well-structured, and realistic; all components
    integrate into a unified governance plan; writing is clear and
    precise; demonstrates thorough command of course concepts across
    the full framework.
    &
    Framework is mostly coherent and feasible; components are largely
    integrated; writing is clear with minor lapses; good command of
    course concepts throughout.
    &
    Framework is partially coherent; some components feel disconnected
    or underdeveloped; writing is adequate; course concepts applied
    unevenly.
    &
    Framework is incoherent, infeasible, or largely incomplete;
    components do not integrate; writing is unclear; limited evidence
    of course concept mastery.
    \\

    \end{longtable}

    } % end \footnotesize

\end{landscape}