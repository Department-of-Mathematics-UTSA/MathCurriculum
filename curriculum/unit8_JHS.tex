%%%%%%%%%%%%%%%%%%%%%%%%%%%%%%%%%%%%%%%%%%%%%%%%%%%%%%%%%%%%%%%%%
\chapter{Unit 8: Jackson Heart Study}
\label{chap:unit8}
%%%%%%%%%%%%%%%%%%%%%%%%%%%%%%%%%%%%%%%%%%%%%%%%%%%%%%%%%%%%%%%%%

\textbf{Time: 3 hours}

%================================================================
\section{Introduction to the Jackson Heart Study}
\label{sec:JHS}
%================================================================

This lesson provides an overview of the Jackson Heart Study (JHS), focusing on its design, data collection, and variable interpretation. Students will describe the JHS's purpose, population, and exam structure, summarize clinical, survey, and genetic data collection methods across study phases, and learn to use JHS codebooks to identify variables for research questions.

\subsection{Learning Objectives}

\begin{enumerate}
    \item Describe the JHS Study Design: Explain the purpose, population, and structure of the JHS, including its major exams.
    \item Summarize Data Collection Methods: Identify the types of data collected (e.g., clinical, survey, genetic) and the methods used in different study phases.
    \item Interpret Key Variables and Codebooks: Understand how to use JHS codebooks to find variables relevant to specific research questions.
\end{enumerate}

\subsection{Assessment Instrument}

\begin{enumerate}
    \item Describe in your own words the purpose of the JHS, cohort characteristics, and exam waves.
    \item Describe how JHS collects specific data (e.g., CAC scores, lipid tests, etc.) and potential biases in data collection.
    \item Answer the following question: ``Association between hysterectomy and cardiovascular disease in the JHS, adjusting for covariates'' by locating relevant variables in the JHS codebook. Describe the variables chosen and their rationale.
\end{enumerate}

%================================================================
\section{The Process of Manuscript Development in the Jackson Heart Study}
\label{sec:JHS-Manuscript}
%================================================================
This lesson covers the process for requesting and obtaining JHS data. Students will learn to navigate data access procedures, including submitting manuscript or ancillary study proposals, completing data use agreements, and addressing ethical considerations to ensure responsible use of JHS data.

\subsection{Learning Objectives}
Explain Data Access Procedures: Describe the process for requesting and obtaining JHS data, including data use agreements and ethical considerations.

\subsection{Assessment}
\begin{enumerate}
    \item In your own words, enumerate the steps involved in the process for developing a JHS manuscript.
    \item Using the information acquired from the lecture, draft a mock JHS manuscript proposal, using the Manuscript Proposal Form provided and the sample manuscript proposal.
\end{enumerate}

% Required packages (add to your preamble if not already there):
% \usepackage{longtable}
% \usepackage{booktabs}
% \usepackage{ragged2e}
% \usepackage{array}
% \usepackage{pdflscape}

\begin{landscape}

    %================================================================
    \section{Evaluation Rubric}
    %================================================================

    {\footnotesize

    \begin{longtable}{%
        >{\RaggedRight\arraybackslash}p{2.5cm}   % Component
        >{\RaggedRight\arraybackslash}p{4.5cm}   % Excellent
        >{\RaggedRight\arraybackslash}p{4.2cm}   % Good
        >{\RaggedRight\arraybackslash}p{4.2cm}   % Satisfactory
        >{\RaggedRight\arraybackslash}p{4.5cm}}  % Needs Improvement

    \toprule
    \textbf{Component} &
    \textbf{Excellent (90--100\%)} &
    \textbf{Good (80--89\%)} &
    \textbf{Satisfactory (70--79\%)} &
    \textbf{Needs Improvement ($<$70\%)} \\
    \midrule
    \endfirsthead

    \toprule
    \textbf{Component} &
    \textbf{Excellent (90--100\%)} &
    \textbf{Good (80--89\%)} &
    \textbf{Satisfactory (70--79\%)} &
    \textbf{Needs Improvement ($<$70\%)} \\
    \midrule
    \endhead

    \midrule
    \multicolumn{5}{r}{\textit{Continued on next page}} \\
    \endfoot

    \bottomrule
    \endlastfoot

    % Row 1
    \textbf{1.\ Variable Identification}
    \newline\textbf{(50\%)}
    &
    Accurately identifies variables; clear rationale; demonstrates deep
    understanding of JHS datasets.
    &
    Minor omissions but overall strong alignment.
    &
    Adequate but limited justification.
    &
    Incomplete or incorrect variable selection.
    \\
    \midrule

    % Row 2
    \textbf{2.\ Manuscript Proposal}
    \newline\textbf{(50\%)}
    &
    Logically extends grant proposal; well-articulated and publication-ready.
    &
    Solid but needs refinement in framing or methods.
    &
    Basic understanding but lacks clarity or completeness.
    &
    Unclear, inconsistent, or incomplete proposal.
    \\
    \midrule

    % Row 3
    \textbf{3.\ Overall Professionalism and Clarity}
    &
    Polished writing, correct formatting, adherence to instructions.
    &
    Minor issues with organization or formatting.
    &
    Understandable but contains stylistic or format errors.
    &
    Poor structure, numerous errors.
    \\

    \end{longtable}

    } % end \footnotesize

\end{landscape}