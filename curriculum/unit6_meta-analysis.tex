%%%%%%%%%%%%%%%%%%%%%%%%%%%%%%%%%%%%%%%%%%%%%%%%%%%%%%%%%%%%%%%%%
\chapter{Unit 6: Meta-analysis}
\label{chap:unit6}
%%%%%%%%%%%%%%%%%%%%%%%%%%%%%%%%%%%%%%%%%%%%%%%%%%%%%%%%%%%%%%%%%

\textbf{Total Time: 3.5 hours}

%================================================================
\section{Key Concepts in Research Synthesis}
\label{sec:6.1}
%================================================================

This lesson provides an introduction to meta-analysis as a tool for quantitative research synthesis, with the initial goal being to estimate robust consensus effects and statistical significance levels across methodologically diverse but independent research studies. We will focus on the setting of population health science research, using data sources with a longitudinal and temporal structure. We will consider how independent data reflecting similar or identical outcomes on different populations, potentially collected with disparate measurement methods and/or varying sampling designs can be integrated to improve precision and to uncover heterogeneity and effect modifiers.

\subsection{Learning Objectives}

\begin{enumerate}
    \item Learners will understand how meta-analysis can be seen as a form of evidence combination.
    \item Learners will be able to apply methods for evidence integration when complete data are available.
    \item Learners will understand the basic mathematics behind pooling p-values, standard errors, and confidence intervals from independent sources.
    \item Learners will be able to weight estimates with different precisions to produce an optimal pooled estimate.
    \item Learners will be exposed to newer approaches for pooling evidence including E-values and empirical likelihood methods.
\end{enumerate}

\subsection{Assessment Instrument}

The assessment will be centered on birth outcomes (pre-term birth, underweight birth, and infant mortality), in the United States and internationally, using official statistics or other sources of population-level data. Learners will be guided to appropriate sources of data, and will begin by estimating event rates at local spatial and temporal scales, and providing uncertainty assessments for all point estimates. For the purposes of this assessment, we will treat evidence sources as independent and homogeneous, even if this is unlikely to be the case.

%================================================================
\section{Accounting for Heterogeneity}
\label{sec:6.2}
%================================================================

Students will learn to separate effect size variability into statistical noise (imprecision) and true heterogeneity. We will discuss several summary measures for unexplained effect heterogeneity, and discuss how stratification and regression methods can be used to understand heterogeneity that can be attributed to known factors.

\subsection{Learning Objectives}

\begin{enumerate}
    \item Students will understand how variation in measured effect sizes can be partitioned into the component attributable to statistical variation, and the component attributable to heterogeneity in the true effects.
    \item Students will understand how the level of heterogeneity in true effects can be partitioned into a component that is attributable to known factors, and a component that is unexplained by known factors.
    \item Students will be able to articulate the difference between statistical uncertainty and effect heterogeneity.
    \item Students will be able to utilize stratification and regression to estimate a consensus effect size and significance level from studies with heterogeneous designs and/or effect heterogeneity.
\end{enumerate}

\subsection{Assessment Instrument}

Building on the work done for Section~\ref{sec:6.1}, learners will be asked to consider how and why spatially and temporally local point estimates may be heterogeneous, and to quantify this heterogeneity. They will then identify scales at which partial pooling may be appropriate, calculate statistically efficient pooled estimates of evidence, and assess the extent to which precision was improved. Finally, they will be asked to identify potential explanatory factors for any observed heterogeneity, and to quantify the extent to which the heterogeneity can be attributed to these factors.

%================================================================
\section{Accounting for Non-independence and Network Effects}
\label{sec:6.3}
%================================================================

Grounded in the setting of population health outcomes assessed at different spatial and temporal scales, we will discuss how and why disparate sources of statistical evidence may be non-independent, and how this non-independence presents both obstacles and possibilities.

\subsection{Learning Objectives}

\begin{enumerate}
    \item Learners will understand how non-independence of research results impacts research synthesis in terms of bias, uncertainty, and statistical power, and will be able to identify possible sources of non-independence.
    \item Learners will revisit the task of combining p-values, Z-scores and other sources of summary evidence from Section~\ref{sec:6.1}, considering extensions that accommodate non-independent evidence measures.
    \item Learners will be able to employ marginal and multilevel regression techniques to account for and leverage the presence of non-independence.
    \item Learners will be exposed to modern methods of evidence summarization such as E-values that are robust to non-independence.
\end{enumerate}

\subsection{Assessment Instrument}

Continuing the work from Sections~\ref{sec:6.1} and~\ref{sec:6.2}, students will reconsider the precision of partially pooled estimates of event rates in light of possible non-independence. Then they will consider whether and how explained and unexplained heterogeneity should be re-evaluated if the evidence sources are non-independent.

% Required packages (add to your preamble if not already there):
% \usepackage{longtable}
% \usepackage{booktabs}
% \usepackage{ragged2e}
% \usepackage{array}
% \usepackage{pdflscape}

\begin{landscape}

    \section{Evaluation Rubric}

    {\footnotesize

    \begin{longtable}{%
        >{\RaggedRight\arraybackslash}p{2.5cm}   % Component
        >{\RaggedRight\arraybackslash}p{4.5cm}   % Excellent
        >{\RaggedRight\arraybackslash}p{4.2cm}   % Good
        >{\RaggedRight\arraybackslash}p{4.2cm}   % Satisfactory
        >{\RaggedRight\arraybackslash}p{4.5cm}}  % Needs Improvement

    \toprule
    \textbf{Component} &
    \textbf{Excellent (90--100\%)} &
    \textbf{Good (80--89\%)} &
    \textbf{Satisfactory (70--79\%)} &
    \textbf{Needs Improvement ($<$70\%)} \\
    \midrule
    \endfirsthead

    \toprule
    \textbf{Component} &
    \textbf{Excellent (90--100\%)} &
    \textbf{Good (80--89\%)} &
    \textbf{Satisfactory (70--79\%)} &
    \textbf{Needs Improvement ($<$70\%)} \\
    \midrule
    \endhead

    \midrule
    \multicolumn{5}{r}{\textit{Continued on next page}} \\
    \endfoot

    \bottomrule
    \endlastfoot

    % Row 1
    \textbf{1.\ Evidence Combination \& Pooled Estimation}
    \newline\textbf{(30\%)}
    &
    Accurately estimates event rates for birth outcomes (pre-term birth,
    underweight birth, infant mortality) at local spatial and temporal scales;
    provides thorough uncertainty assessments for all point estimates; correctly
    pools p-values, standard errors, and confidence intervals from independent
    sources; applies precision-weighted pooling to produce an optimal combined
    estimate; clearly justifies assumptions of independence and homogeneity.
    &
    Estimates event rates with mostly correct uncertainty assessments; pools
    evidence from independent sources with minor errors; weighting approach
    is reasonable with minor gaps in justification.
    &
    Estimates event rates but uncertainty assessments are incomplete or
    partially incorrect; pooling approach is basic or inconsistently applied;
    limited justification for assumptions made.
    &
    Event rate estimates missing or incorrect; uncertainty assessments absent
    or flawed; little to no evidence of understanding pooling methods or
    independence assumptions.
    \\
    \midrule

    % Row 2
    \textbf{2.\ Heterogeneity Quantification \& Partial Pooling}
    \newline\textbf{(35\%)}
    &
    Clearly identifies and quantifies spatial and temporal heterogeneity in
    point estimates; correctly partitions effect size variability into
    statistical noise and true heterogeneity; identifies appropriate scales
    for partial pooling and calculates statistically efficient pooled
    estimates; demonstrates improved precision; identifies and quantifies
    explanatory factors for observed heterogeneity using stratification
    and/or regression.
    &
    Heterogeneity identified and mostly quantified correctly; partial pooling
    applied at reasonable scales with minor errors; some explanatory factors
    identified, though quantification of their contribution may be incomplete.
    &
    Heterogeneity acknowledged but quantification is superficial or
    incomplete; partial pooling attempted but scale selection or efficiency
    poorly justified; explanatory factors noted but not rigorously assessed.
    &
    Heterogeneity not meaningfully addressed; no attempt at partial pooling
    or efficiency assessment; explanatory factors absent or incorrectly
    handled; conflates statistical uncertainty with true heterogeneity.
    \\
    \midrule

    % Row 3
    \textbf{3.\ Non-Independence \& Network Effects}
    \newline\textbf{(35\%)}
    &
    Clearly identifies possible sources of non-independence among evidence
    sources; accurately reassesses precision of partially pooled estimates
    in light of non-independence; correctly revisits heterogeneity findings
    under non-independence; applies marginal or multilevel regression
    techniques appropriately; demonstrates nuanced understanding of how
    non-independence affects bias, uncertainty, and statistical power.
    &
    Sources of non-independence identified with minor gaps; reassessment of
    pooled estimates mostly correct; regression techniques applied with minor
    errors; understands impact on uncertainty and power, though depth may
    be limited.
    &
    Non-independence acknowledged but sources poorly identified; reassessment
    of estimates or heterogeneity is superficial; regression techniques
    attempted but with notable errors or limited justification.
    &
    Non-independence not meaningfully addressed; no reassessment of pooled
    estimates or heterogeneity; regression techniques absent or incorrectly
    applied; little understanding of implications for bias or power.
    \\

    \end{longtable}

    } % end \footnotesize

\end{landscape}